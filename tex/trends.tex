We now seek any other relationships between the explanatory variables and the response variable ppua. Consider first the effect of
the sell month on ppua. Looking at \cref{fig:seasonal} it appears that ppua may fluctuate according to a sinusoidal pattern. But
with that said we have rather limited data, spanning a period of only 12 months and looking at the mean ppua
(\cref{fig:seasonalmean}) does not reveal any pattern whatsoever. So although it is probably still worth exploring how seasonality
affects ppua when fitting our models we should interpret results pertaining to the seasonality with a grain of salt.

At first glance it would appear that the age of a house has a quadratic relationship with ppua. But after separating the data into
urban and rural, we see that both groups show a linear relationship between age and ppua. See \cref{fig:age}. The reason that when
viewed together the data may have appeared quadratic is because rural homes decay in ppua faster than urban homes do with respect
to age, so the lines intersect to form a $v$ shape.

Yet again the initial look is deceptive when looking at \cref{fig:mrt} which appears exponential in nature. However once divided
into rural and urban groups we have a linear relationship with a small slope for urban homes, but still what seems like an
exponential relationship for rural homes. One possible explanation for this difference is that the shape really is exponential but
urban areas will have more MRT stations than rural areas over a smaller footprint and so most homes will be closer to an MRT
station and over this range of distances the relationship looks linear. Some rural homes may happen to be close to an MRT station
but most of them will be farther away spread out across a larger area, allowing us capture the full shape. The fact that we took
the log-transform of ppua does in fact help with this.

When urban and rural properties are taken together there is a linear trend of moderate positive slope between the number of nearby
convenience stores and ppua, see \cref{fig:stores}. When separated, the relationship becomes almost a constant one for urban homes
while for rural homes the linear trend becomes more pronounced. This makes sense for similar reasons as the results about MRT
stations made sense. Notice however that even for rural houses there is little difference between the ppua of houses with between
2 and 4 convenience stores. This could be explained by the fact that being close to 3 stores isn't practically any better than
being close just 2 stores. Distance to the closest store may be a more useful feature. Perhaps the jump in ppua seen in
rural home after being close to 5 or more stores is an artifact of having little data on these types of houses.

The fact that patterns keep emerging in the data when we split into the urban and rural categories is a good sign that we did a
good job of location characterization. It also worth noting that there are no obvious relationships between any two of the
explanatory variables.
