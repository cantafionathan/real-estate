Our analysis of the real estate valuation data set revealed that the growth rate of property values is higher in urban areas
compared to rural areas. It's worth mentioning that although the response variable is simply price per unit area, most of the
explanatory variables in our model are interacting with the date variable allowing us to capture growth in real estate value over
time. This finding suggests that investing in real estate in urban can yield a higher return on investment in the long run.
Furthermore, our analysis showed that properties located closer to public transit systems in urban areas experience higher growth
rates compared to those that are farther away. This observation could be attributed to the convenience and accessibility provided
by public transit systems, making it easier for people to access job opportunities and amenities, thus increasing the demand for
properties in those areas. Therefore, investing in properties close to public transit systems in urban areas could be a viable
strategy for real estate investors looking to maximize their returns. On the other hand, we also found that distance-related
features such as distance to the nearest MRT station, distance to the nearest convenience store, and distance to the city center,
were less important in rural areas compared to urban areas. This may be because people living in rural areas are more likely to
have their own cars and rely less on public transportation. Thus, the demand for properties in rural areas may be influenced by
other factors such as the availability of natural resources, access to good schools, and quality of life. Overall, our analysis
highlights the importance of location and accessibility to public transportation in the real estate market. Investing in urban
areas close to public transit systems could offer higher returns, while investing in rural areas may require a different set of
consideration. 
