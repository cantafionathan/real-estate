Since we are interested in the difference between rural and urban homes, we need to be able to classify homes as either rural or
urban. By using prominent dividing features of the district like rivers and major roads we divided the location into two regions,
urban and rural as seen in \cref{fig:regions}. We now consider the distributions of ppua (price per unit area) for rural and urban
homes separately. Looking at \cref{fig:boxplot} we see that the homes with the highest ppua are predominately urban while still
being atypical compared to other urban homes. This \textit{could} be explained by noting that certain types of properties, such as
luxury apartments which have high ppua, are expected to be more common in urban neighborhoods than rural ones while still being
relatively uncommon overall. Regardless of the explanation, it makes sense to transform ppua by taking the logarithm so that these
outlier values don't influence our models and interpretations disproportionately. For the remainder of this paper, ppua will refer
to log-transformed ppua. As a final note, the ppua of urban homes appears to be noticeably higher on average than that of rural
homes as seen in \cref{fig:ppuamap}.