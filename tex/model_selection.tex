To determine the optimal number of variables to include in our model, we performed both forward and backward selection to
determine which variable count range consistently performed well on different tests. We tested using three criteria: Mallow`s Cp,
Bayesian Information Criteria, and the Adjusted R-Squared. A benefit of using forward/backward selection is that potential
colinearity between features will be implicitly handled since redundant features get discarded.

The best variable count according to Mallow`s Cp is where the variable count is both close to the Mallow`s Cp value and is small.
Consulting \cref{fig:mallowcp} we see that with forward selection a variable count of 9 as the smallest Cp value and its
Cp value is 8.605 which is relatively close to 9, and with backward selection a variable count of 9 is the best because it
has a Cp value of 8.542. Although a variable count of 10-15 have smaller Cp values for backward selection, their Cp values are not
very close to their respective variable counts which makes these variable counts not as optimal. Both forward and backward
selection suggest that 9 variable counts is optimal.

The best variable count according to BIC is the variable count with the smallest BIC value. According to \cref{fig:bic}, with
forward selection a variable count of 7 has the smallest BIC. While with backward selection a variable count of 8 has the smallest
BIC value. In general, a variable count of 7-8 seems to be optimal for the BIC test.

The best value for Adjusted R-Squared is the variable count with the largest Adjusted R-Squared value. Consulting \cref{fig:rsq}
with forward selection a variable count of 17 has the largest Adjusted RSQ. Conversely, with backward selection a variable count
of 14 has the largest Adjusted RSQ. Since forward and backward selection yielded different results, it is inconclusive whether 14
or 17 is the best variable count. According to both figures, it appears that any variable count greater than 8 seems to have a
good adjusted R-Squared.

Overall, we would say that a variable count of 7-9 seems to consistently produce a good result according to the three criterias.
To pick a variable count which is sound, we performed cross validation with variable counts of 7-9 and selected the
variable count with the lowest Root Mean Squared Error (RMSE). After performing the cross validation, a variable count of 9 is selected.
The summary for our best model is shown in \cref{fig:summary}.
